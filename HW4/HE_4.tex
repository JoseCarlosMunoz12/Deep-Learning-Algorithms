\documentclass[12pt,english]{article}
\usepackage[a4paper,bindingoffset=0.2in,%
            left=1in,right=1in,top=1in,bottom=1in,%
            footskip=.25in]{geometry}
\usepackage{blindtext}
\usepackage{titling}
\usepackage{amssymb}
\usepackage{listofitems,amsmath}
\usepackage{listings}
\usepackage{lettrine} 
\usepackage{tikz}  
\usepackage{color} 
 \usetikzlibrary{shapes, arrows, calc, arrows.meta, fit, positioning} % these are the parameters passed to the library to create the node graphs  
\tikzset{  
    -Latex,auto,node distance =0.6 cm and 1.3 cm, thick,% node distance is the distance between one node to other, where 1.5cm is the length of the edge between the nodes  
    state/.style ={ellipse, draw, minimum width = 0.9 cm}, % the minimum width is the width of the ellipse, which is the size of the shape of vertex in the node graph  
    point/.style = {circle, draw, inner sep=0.18cm, fill, node contents={}},  
    bidirected/.style={Latex-Latex,dashed}, % it is the edge having two directions  
    el/.style = {inner sep=2.5pt, align=right, sloped}  
}  
\setlength{\parskip}{12pt}
%================================
\begin{document}
\newgeometry{left=0.8in,right=0.8in,top=1in,bottom=1in}
\begin{center}
    \Large
    \textbf{Homework 4}\\
    \small
    \today\\
    \large
    Jos\'{e} Carlos Mu\~{n}oz
\end{center}
\section*{1}
The original 1D data set is
\begin{align*}
\begin{bmatrix} 2 & 1 & 3 & 4 & 7\end{bmatrix}
\end{align*}
The filter we are using is
\begin{align*}
\begin{bmatrix} 1 & 0 & 1\end{bmatrix}
\end{align*}
Using the convolution the math looks like this
\begin{align*}
 2 * 1 + 1 * 0 + 3 * 1 & = 5\\
 1 * 1 + 3 * 0 + 4 * 1 & = 5\\
 3 * 1 + 4 * 0 + 7 * 1 & = 10
\end{align*}
Therefore, the final matrix will be
\begin{align*}
\begin{bmatrix} 5 & 5 & 10\end{bmatrix}
\end{align*}
\section*{4}
The Convolution Layer will not change the Hieght and width of the incoming matrix. but it will change the depth amount to however there is in that Convolution layer. The MaxPooling used here will only reduce the width and hieght by a factor of 2.\\
Using what we know we can now determine the dimensions of the Tensor as it pases through eacho fhte different Convoluted Neural Networks.\\
For the First CNN
\begin{equation*}
\begin{array}{c|c}
 \hbox{Layer}& \hbox{output of Layer}  \\
\hline
L_1 & 224x224x64 \\
M & 112x112x64 \\
L_2 & 112x112x128 \\
M & 56x56x128 \\
L_3,L_4 & 56x56x256 \\
M & 28x28x256\\
L_5,L_6 & 28x28x512 \\
M & 14x14x512\\
L_7,L_8 & 14x14x512 \\
M & 7x7x512\\
L_9,L_{10} & 14x14x4096 \\
L_{11} & 14x14x1000 
\end{array} 
\end{equation*}
For the Second CNN
\begin{equation*}
\begin{array}{c|c}
 \hbox{Layer}& \hbox{output of Layer}  \\
\hline
L_1 & 224x224x64 \\
M & 112x112x64 \\
L_2 & 112x112x128 \\
M & 56x56x128 \\
L_3,L_4 & 56x56x256 \\
M & 28x28x256\\
L_5,L_6 & 28x28x512 \\
M & 14x14x512\\
L_7,L_8 & 14x14x512 \\
M & 7x7x512\\
L_9,L_{10} & 14x14x4096 \\
L_{11} & 14x14x1000 
\end{array} 
\end{equation*}
\section*{7}
The 7x7 matrix is
\begin{align*}
\begin{bmatrix}6&3&4&4&5&0&3\\4&7&4&0&4&0&4\\7&0&2&3&4&5&2\\3&7&5&0&3&0&7\\5&8&1&2&5&4&2\\8&0&1&0&6&0&0\\6&4&1&3&0&4&5\end{bmatrix}
\end{align*}
The 3x3 matrix
\begin{align*}
\begin{bmatrix} 1 & 1 & 1\\ 0 & 0 & 0\\ -1 & -1 & -1\end{bmatrix}
\end{align*}
Using the convolution the math looks like this, only the first 3 columns of the first row of the new matrix will be calculated
\begin{align*}
4 &= 6*1 + 3*1 + 4*1 + 4*0 + 7*0 + 4*0 + 7*-1 + 0*-1 + 2 *-1\\
3 &= 3*1 + 4*1 + 4*1 + 7*0 + 4*0 + 0*0 + 0*-1 + 2*-1 + 3 *-1\\
4 &= 4*1 + 4*1 + 5*1 + 4*0 + 0*0 + 4*0 + 2*-1 + 3*-1 + 4 *-1
\end{align*}
The final convolution matrix is
\begin{align*}
\begin{bmatrix} 4 & 3 & 4 & -3 & -3\\ 0 & -1 & 0 & 1 & -2\\ -5 & -6 & 1 & -1 & 0\\ 6 & 11 & 1 & -3 & 1\\ 3 & 3 & 4 & 4 & 2\end{bmatrix}
\end{align*}
\end{document}