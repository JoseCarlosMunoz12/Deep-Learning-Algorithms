\documentclass[12pt,english]{article}
\usepackage[a4paper,bindingoffset=0.2in,%
            left=1in,right=1in,top=1in,bottom=1in,%
            footskip=.25in]{geometry}
\usepackage{blindtext}
\usepackage{titling}
\usepackage{amssymb}
\usepackage{listofitems,amsmath}
\usepackage{listings}
\usepackage{lettrine} 
\usepackage{tikz}  
\usepackage{color} 
 \usetikzlibrary{shapes, arrows, calc, arrows.meta, fit, positioning} % these are the parameters passed to the library to create the node graphs  
\tikzset{  
    -Latex,auto,node distance =0.6 cm and 1.3 cm, thick,% node distance is the distance between one node to other, where 1.5cm is the length of the edge between the nodes  
    state/.style ={ellipse, draw, minimum width = 0.9 cm}, % the minimum width is the width of the ellipse, which is the size of the shape of vertex in the node graph  
    point/.style = {circle, draw, inner sep=0.18cm, fill, node contents={}},  
    bidirected/.style={Latex-Latex,dashed}, % it is the edge having two directions  
    el/.style = {inner sep=2.5pt, align=right, sloped}  
}  
\setlength{\parskip}{12pt}
%================================
\begin{document}
\newgeometry{left=0.8in,right=0.8in,top=1in,bottom=1in}
\begin{center}
    \Large
    \textbf{Homework 4}\\
    \small
    \today\\
    \large
    Jos\'{e} Carlos Mu\~{n}oz
\end{center}
\section*{1}
The original 1D data set is
\begin{align*}
\begin{bmatrix} 2 & 1 & 3 & 4 & 7\end{bmatrix}
\end{align*}
The filter we are using is
\begin{align*}
\begin{bmatrix} 1 & 0 & 1\end{bmatrix}
\end{align*}
Using the convolution the math looks like this
\begin{align*}
 2 * 1 + 1 * 0 + 3 * 1 & = 5\\
 1 * 1 + 3 * 0 + 4 * 1 & = 5\\
 3 * 1 + 4 * 0 + 7 * 1 & = 10
\end{align*}
Therefore, the final matrix will be
\begin{align*}
\begin{bmatrix} 5 & 5 & 10\end{bmatrix}
\end{align*}
\section*{4}
The size of the convolution outout can be calculate with the equation $[(W - K + 2*P)/S + 1$. Where W is the dimension of the  incoming tensor. K is the filter size, p is the padding and S is the size of the stride. For these CNN the padding is 1 and the stride is 1. The new tensor size will also have the depth to be same as how many filters therei in the  The MaxPooling for all of these CNN. cuts the dimension in half but keeps the depth the same range.\\
Using what we know we can now determine the dimensions of the Tensor as it pases through eacho fhte different Convoluted Neural Networks.\\
For the First CNN
\begin{equation*}
\begin{array}{c|c}
 \hbox{Layer}& \hbox{output dimensions of Layer}  \\
\hline
L_1 & 224x224x64 \\
M & 112x112x64 \\
L_2 & 112x112x128 \\
M & 56x56x128 \\
L_3,L_4 & 56x56x256 \\
M & 28x28x256\\
L_5,L_6 & 28x28x512 \\
M & 14x14x512\\
L_7,L_8 & 14x14x512 \\
M & 7x7x512\\
L_9,L_{10} & 1x1x4096 \\
L_{11} & 1x1x1000 
\end{array} 
\end{equation*}
For the Second CNN
\begin{equation*}
\begin{array}{c|c}
 \hbox{Layer}& \hbox{output dimensions of Layer}  \\
\hline
L_1 & 224x224x64 \\
LRN & 224x224x64 \\
M & 112x112x64 \\
L_2 & 112x112x128 \\
M & 56x56x128 \\
L_3,L_4 & 56x56x256 \\
M & 28x28x256\\
L_5,L_6 & 28x28x512 \\
M & 14x14x512\\
L_7,L_8 & 14x14x512 \\
M & 7x7x512\\
L_9,L_{10} & 1x1x4096 \\
L_{11} & 1x1x1000 
\end{array} 
\end{equation*}
For the Third CNN
\begin{equation*}
\begin{array}{c|c}
 \hbox{Layer}& \hbox{output dimensions of Layer}  \\
\hline
L_1,L_2 & 224x224x64 \\
M & 112x112x64 \\
L_3,L_4 & 112x112x128 \\
M & 56x56x128 \\
L_5,L_6 & 56x56x256 \\
M & 28x28x256\\
L_7,L_8 & 28x28x512 \\
M & 14x14x512\\
L_9,L_{10} & 14x14x512 \\
M & 7x7x512\\
L_{11},L_{12} & 1x1x4096 \\
L_{13} & 1x1x1000 
\end{array} 
\end{equation*}
For the Fourth CNN
\begin{equation*}
\begin{array}{c|c}
 \hbox{Layer}& \hbox{output dimensions of Layer}  \\
\hline
L_1,L_2 & 224x224x64 \\
M & 112x112x64 \\
L_3,L_4 & 112x112x128 \\
M & 56x56x128 \\
L_5,L_6,l_7 & 56x56x256 \\
M & 28x28x256\\
L_8,L_9,L_{10} & 28x28x512 \\
M & 14x14x512\\
L_{11},L_{12},L_{13} & 14x14x512 \\
M & 7x7x512\\
L_{14},L_{15} & 1x1x4096 \\
L_{16} & 1x1x1000 
\end{array} 
\end{equation*}
For the Fifth CNN
\begin{equation*}
\begin{array}{c|c}
 \hbox{Layer}& \hbox{output dimensions of Layer}  \\
\hline
L_1,L_2 & 224x224x64 \\
M & 112x112x64 \\
L_3,L_4 & 112x112x128 \\
M & 56x56x128 \\
L_5,L_6,l_7 & 56x56x256 \\
M & 28x28x256\\
L_8,L_9,L_{10} & 28x28x512 \\
M & 14x14x512\\
L_{11},L_{12},L_{13} & 14x14x512 \\
M & 7x7x512\\
L_{14},L_{15} & 1x1x4096 \\
L_{16} & 1x1x1000 
\end{array} 
\end{equation*}
For the Sixth CNN
\begin{equation*}
\begin{array}{c|c}
 \hbox{Layer}& \hbox{output dimensions of Layer}  \\
\hline
L_1,L_2 & 224x224x64 \\
M & 112x112x64 \\
L_3,L_4 & 112x112x128 \\
M & 56x56x128 \\
L_5,L_6,L_7, L_8 & 56x56x256 \\
M & 28x28x256\\
L_9,L_{10},L_{11}, L_{12} & 28x28x512 \\
M & 14x14x512\\
L_{13},L_{14},L_{15}, L_{16} & 14x14x512 \\
M & 7x7x512\\
L_{17},L_{18} & 1x1x4096 \\
L_{19} & 1x1x1000 
\end{array} 
\end{equation*}
\section*{7}
The 7x7 matrix is
\begin{align*}
\begin{bmatrix}6&3&4&4&5&0&3\\4&7&4&0&4&0&4\\7&0&2&3&4&5&2\\3&7&5&0&3&0&7\\5&8&1&2&5&4&2\\8&0&1&0&6&0&0\\6&4&1&3&0&4&5\end{bmatrix}
\end{align*}
The 3x3 matrix
\begin{align*}
\begin{bmatrix} 1 & 1 & 1\\ 0 & 0 & 0\\ -1 & -1 & -1\end{bmatrix}
\end{align*}
Using the convolution the math looks like this, only the first 3 columns of the first row of the new matrix will be calculated
\begin{align*}
4 &= 6*1 + 3*1 + 4*1 + 4*0 + 7*0 + 4*0 + 7*-1 + 0*-1 + 2 *-1\\
3 &= 3*1 + 4*1 + 4*1 + 7*0 + 4*0 + 0*0 + 0*-1 + 2*-1 + 3 *-1\\
4 &= 4*1 + 4*1 + 5*1 + 4*0 + 0*0 + 4*0 + 2*-1 + 3*-1 + 4 *-1
\end{align*}
The final convolution matrix is
\begin{align*}
\begin{bmatrix} 4 & 3 & 4 & -3 & -3\\ 0 & -1 & 0 & 1 & -2\\ -5 & -6 & 1 & -1 & 0\\ 6 & 11 & 1 & -3 & 1\\ 3 & 3 & 4 & 4 & 2\end{bmatrix}
\end{align*}
\end{document}
