\documentclass[12pt,english]{article}
\usepackage[a4paper,bindingoffset=0.2in,%
            left=1in,right=1in,top=1in,bottom=1in,%
            footskip=.25in]{geometry}
\usepackage{blindtext}
\usepackage{titling}
\usepackage{amssymb}
\usepackage{amsmath}
\usepackage{listings}
\usepackage{lettrine} 
\usepackage{tikz}  
\usepackage{color} 
 \usetikzlibrary{shapes, arrows, calc, arrows.meta, fit, positioning} % these are the parameters passed to the library to create the node graphs  
\tikzset{  
    -Latex,auto,node distance =0.6 cm and 1.3 cm, thick,% node distance is the distance between one node to other, where 1.5cm is the length of the edge between the nodes  
    state/.style ={ellipse, draw, minimum width = 0.9 cm}, % the minimum width is the width of the ellipse, which is the size of the shape of vertex in the node graph  
    point/.style = {circle, draw, inner sep=0.18cm, fill, node contents={}},  
    bidirected/.style={Latex-Latex,dashed}, % it is the edge having two directions  
    el/.style = {inner sep=2.5pt, align=right, sloped}  
}  
\setlength{\parskip}{12pt}
\title{Home Work 1 Undergraduate}
\date{\today}
\author{Jose Carlos Munoz}
%================================
\begin{document}
\newgeometry{left=0.8in,right=0.8in,top=1in,bottom=1in}
\begin{center}
    \Large
    \textbf{Homework 1}\\
    \small
    \today\\
    \large
    Jose Carlos Munoz
\end{center}
%===============================
ex1)\\
We assume that in this Neural Network that there is a hidden layer and an output layer. The Hidden layer will contain 2 nodes used to get a preactivation values. The ReLU will be used to get the representation of \\
ex4)\\
\begin{equation}
\begin{array}{c|cc|c}
 & X_1 &  X_2  & Y\\
\hline
a_0 &-1 & -1 & -1\\
\hline
a_1& 1 &  1  & -1\\
\hline
a_2 & 1 & -1 & 1\\
\hline
a_3& -1 & 1  & 1\\
\end{array}
\end{equation}
The Starting $\vec{w}$ is $\{0,0\}$ with an $\alpha = 1$\\
Step 1)
\begin{equation}\tag{1}
\begin{split}
\vec{w}_{1} &= \vec{w}_0+ \alpha * (-1) *a_0\\
&= \{1,1\}
\end{split}
\end{equation}
Step 2)
\begin{equation}\tag{1}
\begin{split}
\vec{w}_{2} &= \vec{w}_1+ \alpha * (-1) *a_1\\
&= \{2,2\}
\end{split}
\end{equation}
Step 3)\\
\begin{equation}\tag{3}
\begin{split}
\vec{w}_{3} &= \vec{w}_2+ \alpha * (1) *a_2\\
&= \{3,1\}
\end{split}
\end{equation}
Step 4)\\
\begin{equation}\tag{4}
\begin{split}
\vec{w}_{4} &= \vec{w}_3+ \alpha * (1) *a_3\\
&= \{2,2\}
\end{split}
\end{equation}
After a few Cycles, we see that it will not converge at all. This is because these points can not be linearly seperated
\end{document}
