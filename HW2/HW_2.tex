\documentclass[12pt,english]{article}
\usepackage[a4paper,bindingoffset=0.2in,%
            left=1in,right=1in,top=1in,bottom=1in,%
            footskip=.25in]{geometry}
%Packages being used in this Tex file
\usepackage{amsmath} % for aligned
\begin{document}
ex2)\\
we know that
\begin{align*}
x_1&=2 & x_2&= 3 & \frac{\delta L}{\delta o}&=5\\
     &     & o   &= x_1 * x_2 
\end{align*}
To find $\frac{\delta L}{\delta x_1}$ and $\frac{\delta L}{\delta x_2}$ we use the Chain rule which gives us $\frac{\delta L}{\delta o}\frac{\delta o}{\delta x_1}$ and $\frac{\delta L}{\delta o}\frac{\delta o}{\delta x_2}$ respectively. It can be derived that $\frac{\delta o}{\delta x_1}$ and $\frac{\delta o}{\delta x_2}$ are $x_2$ and $x_1$ respectively\\
Therefore we can solve for both\\
\begin{align*}
\frac{\delta L}{\delta x_1}&=\frac{\delta L}{\delta o}\frac{\delta o}{\delta x_1} & \frac{\delta L}{\delta x_2}&=\frac{\delta L}{\delta o}\frac{\delta o}{\delta x_2}\\
                                      &=5*x_2  &  &=5 *x_1\\
                                      &=5*3  &  &=5 *2\\
\frac{\delta L}{\delta x_1}&=15  &  \frac{\delta L}{\delta x_2}&=10\\
\end{align*}
hw2)\\
\begin{align*}
  f(x) &= x^2\\
 &= \frac{1}{x}\
\end{align*}
\begin{align*}
  f(x) &= x^2 &  g(x) &= \frac{1}{x}\\
  F(x) &= \int^a_b \frac{1}{3}x^3 &  F(x) &= \int^a_b \frac{1}{3}x^3
\end{align*}

\end{document}